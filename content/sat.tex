\section{Dependency Management through SAT Solvers}

In this section, I'll describe in detail how the new algorithm works. The resolution algorithm consists of two independent algorithms:

\begin{enumerate}
\item An algorithm to construct a local repository based on the transitivity of dependency.
\item An algorithm to use an SAT solver to find an optimal solution on the local repository.
\end{enumerate}

The first algorithm, which I call the \emph{closure construction algorithm}, recursively constructs a repository whose packages are the transitive closure of the initial constraints. In the process of constructing the local repository, it has to handle particular features of Maven and Ivy, such as \emph{fuzzy constraints}, \emph{excludes}, \emph{intransitive dependencies}, \emph{forces}, \emph{scopes} or \emph{configurations}. We need two versions of the algorithm here, one for Maven and one for Ivy.

The second algorithm, which I call the \emph{encoding algorithm}, encodes the local repository to the format of a specific SAT solver, and decodes the result to a user-friendly format. We only need one version of this algorithm.


\subsection{Formalism}

Formalization of the problem based on set semantics is the first step to solve the problem via SAT solvers. The formalism used here is based on the work of Mancinelli et al.\cite{mancinelli2006managing}.

There is a big disparity between the concepts in the Maven ecosystem and the Ivy ecosystem. For example, a package in Maven is very different from a package in Ivy. In Maven, a package is the same as an artifact, but in Ivy a package may define several artifacts. The difference can be abstracted away for the purpose of dependency management based on following two concepts.

\begin{definition}[Library]
  A library is a unique entity which can be versioned.  The versioning of a library should be a total-order. In the Java world, a library is usually identified by the pair $(group, name)$.
\end{definition}

\begin{definition}[Package]
  A package is a versioned library. Two packages of the same library but of different versions conflict.  A package can be denoted as a pair $(l, v)$.
\end{definition}

% \subsubsection{Fuzzy Constraints}

\noindent
\textbf{Fuzzy Constraints}. It's possible to specify fuzzy constraints both in Maven and Ivy. For example, a package $p$ may dependent on $x(\geq 2.3)$. To encode the fuzzy constraints, we can expand $x(\geq 2.3)$ to a set of  concrete packages, such as $\{x_1, x_2, x_3, ...\}$.

Following the approach above, it's possible to encode the direct dependency constraints of any package as a set of sets of packages. This insight enables us to formalize the concept \emph{repository}\cite{mancinelli2006managing}.

\begin{definition}[Repository]
  A repository is a tuple $R = (P, D, C)$ where $P$ is a set of packages, $D : P \rightarrow \mathscr{P p 1}(\mathscr{P p 1}(P))$ is the dependency function\footnote{$\mathscr{P p 1}(X)$ represents the set of subsets of X.}, and $C \subset P \times P$ is the conflict relation. The repository must satisfy the following conditions:

  \begin{itemize}
  \item The relation $C$ is symmetric, i.e., $(\pi_1, \pi_2) \in C$ if and only if $(\pi_2, \pi_1) \in C$ for all $\pi_1, \pi_2 \in C$.
  \item Two packages with the same unit but different versions conflict, i.e., if $\pi_1 = (l, v_1)$ and $\pi_2 = (l, v_2)$ with $v_1 \neq v_2$, then $(\pi_1, \pi_2) \in C$.
  \end{itemize}
\end{definition}

In a repository $R = (P, D, C)$, the dependencies of each package $p$ are given by $D(p) = \{d_1, ..., d_k\}$, which is a set of sets of packages. It means if $p$ is to be used, then all its $k$ dependency requirements must be satisfied. For the constraint $d_i$ to be satisfied, at least one of the packages in $d_i$ must be used as well.

For example, if $D(p) = \{\{a_1, ..., a_k\}, \{c_1, ..., c_j\}\}$, then in order to use $p$, at least one package in $\{a_1, ..., a_k\}$ and one in $\{c_1, ..., c_j\}$ must be used.

Now we can formalize the concept \emph{resolution}.

\begin{definition}[Resolution]
  A resolution for the initial package $\pi$ with regard to a repository $R = (P, D, C)$ is a subset $I$ of $P$. The resolution $I$ must satisfy following conditions:

  \begin{itemize}
  \item \emph{Inclusion}: The initial package is in the resolution. Formally, $\pi \in I$.
  \item \emph{Abundance}: Every package has what it needs. Formally, for every $\pi \in I$, and for every dependency $d \in D(\pi)$ we have $I \cap d \neq \emptyset$.
  \item \emph{Peace}: No two packages conflict. Formally, $(I \times I) \cap C = \emptyset$.
  \end{itemize}
\end{definition}

Note that a resolution may not always exist. There could be conflicts or missing dependencies. In the following, I use \emph{root package} as a synonym of \emph{initial package}, and \emph{root dependencies} means the dependencies of the root package.

\subsection{Closure Construction Algorithm}

The closure construction algorithm is basically a depth-first-search(DFS) algorithm on the \emph{flattened dependency graph} of the closure repository of the root package, which is defined as follows.

\begin{definition}[Flattened Dependency Graph]
  The flattened dependency graph $G = (N, E)$ for a repository $R = (P, D, C)$ is a directed graph, where the nodes are the packages, and the arcs are the dependencies between packages. Formally, $N = P$, and $E = \{ (p, q) | \exists d_i \in D(p) \land q \in d_i\}$.
\end{definition}

Note that there could be cycles in the graph. Following example demonstrates how cycles could arise in practice.

\begin{verbatim}
(x, 2.3) depends on y(>=1.3)
(y, 1.4) depends on x(>=2.1)
\end{verbatim}

It's obvious that in the example above, there is an arc from $(x, 2.3)$ to $(y, 1.4)$, because $(y, 1.4) \in \{(y, v) | v \geq 1.3 \}$. For similar reasons, there is an arc from $(y, 1.4)$ to $(x, 2.3)$. Thus, in this case there is a cycle in the flattened dependency graph. The existence of cycles in the graph implies that the algorithm should keep track of visited nodes in order to avoid infinite loops.

The algorithm is implemented as a recursive function in Scala, which has following signature:
\begin{lstlisting}[language=Scala]
  // Maven
  def resolve(pom: MDescriptor, scope: Scope,
               excludes: Iterable[LibT], path: Set[PackageT]): Unit
  // Ivy
  def resolve(ivy: IDescriptor, confs: Set[String],
               excludes: Seq[IExclude], path: Set[PackageT]): Unit
\end{lstlisting}

The algorithm starts from the root package, and advances following the depth-first-search approach on the flattened dependency graph of the closure repository. In the visiting process, the algorithm updates two mutable data structures $packagesMap$ and $librariesMap$:

\begin{lstlisting}[language=Scala]
  // Maven
  type DependenciesT = Set[(MDependency, Set[MPackage])]

  val packagesMap = new TrieMap[MPackage, DependenciesT]
  val librariesMap = new TrieMap[ILib, Set[MPackage]]

  // Ivy
  type DependenciesT = Set[(IDependency, Set[IPackage])]

  case class PackageInfo(dependencies: DependenciesT, descriptor: IDescriptor,
                          activeConfs: Set[String], activeArtifacts: Set[String])

  val packagesMap = new TrieMap[IPackage, PackageInfo]
  val librariesMap = new TrieMap[ILib, Set[IPackage]]
\end{lstlisting}

Compared to the repository triple $R = (P, D, C)$, the structure $packagesMap$ is the representation of the dependency function $D$, and as $P$ is the domain of $D$, thus is represented by the keys of $packagesMap$. The structure $librariesMap$ is a representation of conflicts $C$. Additional information is maintained in $packagesMap$ for more detailed resolution report.

In the following, I'm going to introduce in detail how the particular features of Maven or Ivy are handled by the \emph{closure construction algorithm}.

\subsubsection{Fuzzy Constraints}

Fuzzy constraints are expanded to individual packages as mentioned before. To handle a fuzzy constraint $L(pred)$, the algorithm first retrieves a list of all versions of the library $L$, then select valid versions by the constraint predicate $pred$.

In Scala, the relevant code snippet is as follows:

\begin{lstlisting}[language=Scala]
  // Maven
  metaResolver(dep.lib).map(dep.filterVersions)

  // Ivy
  versionsResolver(dep.lib).map(dep.filterVersions)
\end{lstlisting}

\subsubsection{Intransitive Constraints}

Ivy supports intransitive dependencies. To support this feature, only advance the depth-first-search if current dependency is transitive. In Scala, we only need to add a conditional check before the recursive call:

\begin{lstlisting}[language=Scala]
  if (dep.transitive)
      resolve(descriptor, depConfs, newExcludes, path + ivy.pkg)
\end{lstlisting}

\subsubsection{Scopes or Configurations}

Maven supports \emph{scopes}, Ivy supports a similar but more flexible concept \emph{configuration}. To support this feature, they are added to the signature of the recursive function:

\begin{lstlisting}[language=Scala]
  // Maven
  def resolve(pom: MDescriptor, scope: Scope, ...): Unit = {
    val deps = pom.filterDependencies(scope, excludes)
    deps.foreach { dep =>
      val newScope = COMPILE
      // ...
    }
  }

  // Ivy
  def resolve(ivy: IDescriptor, confs: Set[String], ...): Unit = {
    val deps = ivy.filterDependencies(confs, excludes)
    deps.foreach { dep =>
      val newConfs = ivy.filterDepConfigurations(confs, dep)
      // ...
    }
  }
\end{lstlisting}

The first line in the body of the function filters the effective dependencies according to the semantics of scope and configurations respectively.

For Maven, the new scope is always \emph{compile}, because Maven scopes are fixed, and it doesn't make sense for one package to depend on scopes other than \emph{compile} of another package.

For Ivy, the new configurations are calculated for each dependency according to the dependency mapping of configurations defined in the descriptor.

\subsubsection{Excludes}

\label{sat:excludes}

Both Maven and Ivy support excludes in dependency. This feature complicates the \emph{closure construction algorithm}. Ideally, each package in the \emph{flatten dependency graph} is visited once. However, the effective scope of excludes is along the transitive chain, it's possible that for a dependency $d$ of a package $\pi$, it's excluded in one chain, but included in another chain. This is illustrated by following example:

\begin{verbatim}
p   depends on(exclude=t)   h(deps={t(>2.1)})
q   depends on   h(deps={t(>2.1)})
\end{verbatim}

In the example above, the dependency path from $p$ to $h$ excludes the library $t$, while the dependency path from $q$ to $h$ includes $t$. This implies we should allow each package to be visited multiple times along different paths, and the dependencies of different paths should be added together by the set union operation. As the combination of paths in a graph could be exponential, the cost in path traversal is potentially exponential. That's one of the reason why I think we should restrict features in artifact specification. I'll discuss more about the topic in section \ref{pragmatics}.

The parameter \emph{path} in the signature represents the visited path. The parameter \emph{excludes} represents the effective excludes.

\begin{lstlisting}[language=Scala]
  // Maven
  def resolve(pom: MDescriptor, scope: Scope,
               excludes: Iterable[LibT], path: Set[PackageT]): Unit
  // Ivy
  def resolve(ivy: IDescriptor, confs: Set[String],
               excludes: Seq[IExclude], path: Set[PackageT]): Unit
\end{lstlisting}

Inside the function body, the excludes are used to filter effective dependencies. Both the \emph{exludes} and \emph{path} are accumulated along the visited path, but different paths don't interfere with each other.

\subsubsection{Overrides}

Ivy supports specifying an override mediation rule, overriding the revision requested for a transitive dependency. This feature can be useful when a direct dependency is bringing a transitive dependency for which the programmers want to change the revision, without actually declaring a dependency on it (because the module doesn't actually depend on it).

Current implementation doesn't support this feature, but technically it's similar to the support of \emph{excludes}, which is also effective and accumulative along the transitive dependency chain.

When interpreting a dependency constraint $L(pred)$, the algorithm should first check if the library $L$ is covered in the \emph{overrides} set. If it's, then the version constraint $pred$ should be replaced by the version constraint specified in the override rules.

In case where there are two override rules on the same library, the override rule closer to the root package has higher precedence\footnote{The Ivy official web site doesn't document how to handle such cases. This is just a reasonable proposition.}.

\subsubsection{Force}

Ivy supports a boolean flag \emph{force} on a dependency constraint, which says that this dependency
should be forced to the specified revision. The effective scope of \emph{force} is global\footnote{Ivy documentation is obscure about this point, but that seems to be the only reasonable interpretation. \\\url{http://ant.apache.org/ivy/history/latest-milestone/ivyfile/dependency.html}} -- instead of being along the transitive dependency chain -- it means if a specific version of a library $L$ is forced, it has to override all dependencies on $L$ for every package in the repository.

The global effective scope of \emph{force} implies we should do post-processing on the dependencies of each package in order to make \emph{force} effective. If there are multiple \emph{forces} on the same library, then it's up to the solver to select one from them -- usually the latest version is chosen\footnote{The Ivy official web site doesn't say anything about the scenario, it's just a reasonable proposition.}.

This feature has not been implemented yet. But given the semantics above, it's not hard to implement this feature by keeping track of all forces during graph traversal, and then do post-update on dependencies of forced libraries for each package.

\subsection{Encoding Algorithm}

Once the local repository has been constructed, the \emph{encoding algorithm} can encode the repository into a SAT problem and call a SAT solver to find a resolution. The encoding algorithm depends on an abstract representation of repository, which has following signature:

\begin{lstlisting}[language=Scala]
  abstract class Repository { outer =>
    type LibT   <: Lib
    type PackageT    <: Package { type LibT = outer.LibT }
    type DependencyT <: Dependency { type LibT = outer.LibT }

    /** Returns the root package of the repository */
    def root: PackageT

    /** Returns the packages that p depends on directly */
    def apply(p: PackageT): Iterable[(DependencyT, Iterable[PackageT])]

    /** Returns all packages in the repository, except the root */
    def packages: Iterable[PackageT]

    /** Returns all primitive conflicts in the repository */
    def conflicts: Map[LibT, Set[PackageT]]
  }
\end{lstlisting}

This abstract interface enables us to use the same encoding algorithm for both Maven and Ivy. In the following I'll explain in detail how to encode the repository into a SAT problem. I'll keep the introduction abstract, so that it's possible to switch to other SAT solvers following the tricks introduced here. For readers interested in the details of encoding the problem in SAT4J, please refer to our code base on Github\footnote{\url{https://github.com/liufengyun/bacala/blob/master/src/main/scala/bacala/alg/SatSolver.scala}}.

The encoding tricks are based on the work of Berre et al. \cite{berre2009dependency}. The implementation is based on SAT4J Pseudo\footnote{\url{http://www.sat4j.org}} version $2.3.4$.

\subsubsection{Encode Dependency}

Remember that in our formalism of a repository $R = (P, D, C)$, the dependency of a package $p$ is like $D(p) = \{\{a_1, ..., a_k\}, \{c_1, ..., c_j\}\}$. It means in order to use $p$, at least one package in $\{a_1, ..., a_k\}$ and one in $\{c_1, ..., c_j\}$ must be used. If we take each package as a proposition variable, this naturally translates to following logical formula:

\[
p \rightarrow ((a_1 \vee ... \vee a_k) \wedge (c_1 \vee ... \vee c_j))
\]

Generally, for a package $p$ with $D(p) = \{ d_1, d_2, ..., d_n\}$ and $d_i = \{ q_{i, j} | i \leq j \leq m_i \}$, it can be encoded as following logical formula:

\[
p \rightarrow \bigwedge_{1 \leq i \leq n} \bigvee_{1 \leq j \leq m_i} q_{i,j}
\]

Some SAT solvers require the input in \emph{conjunctive normal form}, which can be derived as follows:

\[
p \rightarrow \bigwedge_{1 \leq i \leq n} \bigvee_{1 \leq j \leq m_i} q_{i,j}
\quad \quad \equiv \quad \quad
\bigwedge_{1 \leq i \leq n} p \rightarrow \bigvee_{1 \leq j \leq m_i} q_{i,j}
\quad \quad \equiv \quad \quad
\bigwedge_{1 \leq i \leq n} (\neg p \vee \bigvee_{1 \leq j \leq m_i} q_{i,j})
\]

If the package $p$ is the root package, it's always required. Thus the dependency of the root package can be encoded by dropping the implication:

\[
\bigwedge_{1 \leq i \leq n} \bigvee_{1 \leq j \leq m_i} q_{i,j}
\]

\subsubsection{Encode Conflicts}

For given a repository $R = (P, D, C)$, the conflicts $C$ is a set of pairs like $(p, q)$. A natural approach is to encode the conflict pair $(p, q)$ as $\neg p \vee \neg q$. Then all conflicts in the repository can be encoded as following formula:

\[
\bigwedge_{(p, q) \in C} \neg p \vee \neg q
\]

A slightly different encoding is based on the fact that all package conflicts in software development are due to version conflicts on the same library. Suppose the function $CL(L)$ returns all versions of the library $L$ in the repository, then it's possible to encode the conflict on library $L$ as a cardinality constraint\footnote{Cardinality constraint is supported by all major SAT solvers}:

\[
(\sum_{p \in CL(L)} p) \leq 1
\]

In the implementation, the second approach is taken. Unlike the comment in \cite{berre2009dependency}, which claims the second approach is superior for error reporting, we observed no such advantage in our implementation. We take the second approach mainly because it makes logical translation simpler.

\subsubsection{Choose the Optimal Solution}

There could be multiple resolutions for a given repository. In that case, an optimal solution should be chosen. In the context of software package dependency management, new versions of a library are preferred over old versions of a library, so the solver should favor the solution which contains the newest versions of libraries.

SAT4J supports setting a \emph{pseudo-boolean function} to minimize for resolution. The objective function has following form:

\[
\sum_{p \in P} w_p p
\]

So the problem becomes \emph{how to define the weight for each package $p$ in the repository}. The approach we take is rank-based weight. For each library, we rank all versions of the library in the repository, newer versions get lower ranks. Then we use the rank of a package for its weight in the objective function.

This approach works well in our experiment, and it's better than approaches that depend on version numbers, as the latter are prone to version skew, where the library has large version numbers will be dominating in the objective function. The rank-based approach can be seen as having performed a normalization step on the version numbers.

\subsubsection{Report Friendly Error Message}

In case there is no solution for a given repository, the algorithm should report friendly error messages to end users. There are two problems related to user-friendliness.

First, as each library has several candidate versions in the repository, there are many combinations of packages where missing dependencies or conflicts can happen. A user friendly approach is to choose the minimal combination of packages where this is no solution.

Second, as conflicts or missing dependencies can happen very deep in the graph, just reporting these errors will confuse end users, as these packages are not directly required by the root package. A friendly error report should relate the missing dependencies or conflicts to the root dependencies.

From the logical point view, the first problem is equivalent to find the minimal subset $T$ of the total formulas $S$ that cannot be satisfied altogether. This subset of formulas is also called MUS(\emph{minimal unsatisfiable subformula}). Formally, $T$ is MUS of $S$ if and only if $T \subseteq S$, $T \vDash \perp$ and $\nexists R \subset T \wedge R \vDash \perp$.

An elegant algorithm to compute MUS is QuickXplain\cite{junker2004quickxplain}. The algorithm has the merit that it doesn't require any change to the SAT solver. QuickXplain works by translate the constraints $S = \{ s_1, ..., s_n\}$ into $S'$ by adding a new selector variable $sel_i$ to each constraint in $S: S' = \{ sel_1 \vee s_1, ..., sel_n \vee s_n\}$. Then the problem is transformed into a pseudo-boolean problem to minimize the objective function $\sum_{1 \leq i \leq n} sel_i$ under the constraint $S' \bigwedge_{1 \leq i \leq n} \neg sel_i$, which can be readily solved by existent SAT solvers.

To deal with the second requirement, we need to relate each translated logical formula to the original dependency in the repository. When the SAT solver produces MUS, we are sure that the dependencies that correspond to the formulas must form paths from the root package to the place where missing dependencies or conflicts happen. Otherwise, it's possible to show the the set of unsatisfiable subformula is not minimal, which is a contradiction.

Based on this observation, we are able to display the missing dependencies or conflicts as a tree from the root package to the level where the error happens, which makes it clear to end users how the root dependencies lead to the error transitively.

\subsection{Performance}

For the POM and Ivy files we tested, the SAT resolver never takes more than one second. The largest POM file we tested has 46 packages in the solution set, and it took 0.17 second on SAT solver, 68.4 seconds in downloading POM files. With local file system caching of POM files, the cost of network IO drops quickly after the first run on the project.

% When all the POM and Ivy files are cached locally, the new algorithm is about 30 times faster than Maven and Ivy.
